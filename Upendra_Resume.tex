%%%%%%%%%%%%%%%%%%%%%%%%%%%%%%%%%%%%%%%%%
% Medium Length Professional CV
% LaTeX Template
% Version 2.0 (8/5/13)
%
% This template has been downloaded from:
% http://www.LaTeXTemplates.com
%
% Original author:
% Trey Hunner (http://www.treyhunner.com/)
%
% Important note:
% This template requires the resume.cls file to be in the same directory as the
% .tex file. The resume.cls file provides the resume style used for structuring the
% document.
%
%%%%%%%%%%%%%%%%%%%%%%%%%%%%%%%%%%%%%%%%%

%----------------------------------------------------------------------------------------
%	PACKAGES AND OTHER DOCUMENT CONFIGURATIONS 
%----------------------------------------------------------------------------------------
 
\documentclass{resume} % Use the custom resume.cls style 
\usepackage{hyperref}
\usepackage[left=0.4 in,top=0.15 in,right=0.4 in,bottom=0.3in]{geometry} % Document margins
\newcommand{\tab}[1]{\hspace{.2667\textwidth}\rlap{#1}}
\newcommand{\itab}[1]{\hspace{0em}\rlap{#1}}
\name{\large Upendra Kumar Devisetty PhD} % Your name
\address{upendrakumar.devisetty@gmail.com \\ (520)-392-2482 \\ Durham, NC} % Your address 
\address{\href{https://github.com/upendrak}{github.com/upendrak} \\ \href{http://linkedin.com/in/upendradevisetty}{linkedin.com/upendradevisetty} \\ \href{https://scholar.google.com/citations?user=12fLkPsAAAAJ}{Google scholar}} % Your phone number and email
\begin{document}  

%----------------------------------------------------------------------------------------
\begin{rSection}{Skills} \itemsep -3pt  

{\textbf{Languages:} Python, R, SQL, Bash}  \\
{\textbf{Tools:} Pandas, Matplotlib, Seaborn, NumPy, Keras, PySpark, Scikit-learn, Jupyter, Github, Shiny, Flask, BigQuery, Snakemake, MySQL, Postgres, AWS, Ansible, Openstack, Heroku, Docker and familiarity with Kubernetes} \\
{\textbf{Machine Learning:} Data wrangling, Feature engineering, Regression, Classification, Clustering, Decision Trees, Ensemble methods, Convolutional Neural Networks, Recurrent Neural Networks, NLP, Recommendation systems} \\
{\textbf{Techniques:} DevOps, Hadoop, HPC, GPU and distributed computing}
\end{rSection}  
 
\vspace{-7pt}
%-----------------------------------------------------------------------------

\begin{rSection}{Experience } \itemsep -1pt        

\vspace{-3pt}

\begin{rSubsection}{Senior Data Scientist, Greenlight Biosciences, NC }{Jan 2020 - Present}{}    

\vspace{-3pt}

\item Delivering machine learning models, and \textbf{data-driven} insights, that will drive the company's value creation.
\item Next-Generation Sequencing analysis of the genomic \textbf{Big data} to help deliver high-quality targets for \textbf{RNAi}.
\item Setting up the computational infrastructure that meets the demands of the \textbf{Data Science} Team. 
\end{rSubsection} 

\vspace{-7pt}

\begin{rSubsection}{Science Informatician, CyVerse, AZ }{Jan 2016 - Dec 2019}{}    

\vspace{-3pt}

\item Optimized the process of building Bioinformatics and Data Science software into CyVerse using \textbf{Docker}.
\item Developed custom Bioinformatics and computational workflows (MAKER) in the \textbf{Jetstream} cloud for processing complex biological data in a distributed processing environment using \textbf{Work-queue} and \textbf{Pegasus}.
\item Developed novel Bioinformatic pipelines (Eg. RMTA) for mining \textbf{100+ TB} of publicly available RNA-seq data. 
\end{rSubsection} 

\vspace{-7pt}

\begin{rSubsection}{Technical Consultant, Insight Data Science, Seattle, WA}{Sept 2019 - Feb 2020}{}

\vspace{-3pt}

\item Developed and delivered workshops such as Introduction to AWS for Data Scientists, Big Data processing platforms (Hadoop and Spark), Flask web development, ML model deployment using Heroku to Data Engineering fellows.

\vspace{-7pt}

\end{rSubsection}

\begin{rSubsection}{Data Science Fellow, Insight Data Science, Seattle, WA}{June 2019 - Sept 2019}{}

\vspace{-3pt}

\item Built \href{http://plantmd.onrender.com}{PlantMD}, an image-based plant disease detection web app that can rapidly and accurately diagnose plant diseases with \textbf{99\% validation accuracy} and achieving an \textbf{ROC-AUC score of 0.92}.
\item Trained and validated \textbf{Alexnet} and \textbf{VGG16} CNN architectures. Used \textbf{(100K, 500GB)} diseased and healthy plant leaf images using  using \textbf{Keras} on \textbf{Google Collabs} GPU nodes.
\item Used \textbf{Docker}, \textbf{Github} and \textbf{Dockerhub} to automatically manage building and deploying PlantMD on \textbf{AWS}.
\end{rSubsection} 

\vspace{-7pt}

\begin{rSubsection}{Data Science Instructor, Datacamp}
{Jan 2019}{}{}  %June 2016 - November 2016
\vspace{-3pt}
\item Designed and developed course content for \textbf{Big Data Fundamentals via PySpark} using \textbf{Apache PySpark} and its components (RDD, DataFrames, SparkSQL and MLlib). The course has over 8000 students.
\vspace{-7pt}
\end{rSubsection}  

\begin{rSubsection}{Research Associate, Oregon State University, Corvallis}
{Jan 2015 - Dec 2015}{}{}  %June 2016 - November 2016
\vspace{-3pt}

\item Developed cheap and high-throughput DNA extraction and DNA library construction protocols for structural polymorphism discovery in Populus.

\vspace{-7pt}

\end{rSubsection}  

\begin{rSubsection}{Postdoc Researcher, University of California, Davis}
{Mar 2010 - Dec 2014}{}{}  %June 2016 - November 2016
\vspace{-3pt}

\item Detected molecular genetic markers and constructed the novel comprehensive transcriptome assembly pipeline using Brassica rapa RNA-Seq.

\vspace{-7pt}

\end{rSubsection}  

\end{rSection}

%	PROJECTS

\begin{rSection}{PROJECTS}

%------------------------------------------------
\begin{rSubsection}{Automatic classification of research paper abstracts}{Jan 2019}{} %Jun 2016 - Jul 2016   

\vspace{-3pt}
\item Scraped more than \textbf{10k+} research paper abstracts from Google Scholar using an automated custom script and separated abstracts containing the word 'cancer' from those containing the phrase 'machine learning'.
\item Employed \textbf{Logistic Regression (Scikit-learn)} and \textbf{deep neural network (Keras)} to classify research paper abstracts into 'cancer' and 'machine learning' after data cleaning using \textbf{NLTK}. \end{rSubsection} 

\vspace{-7pt}

\begin{rSubsection}{ML-based non-coding RNA detection}{June 2018}{} %Jun 2016 - Jul 2016   

\vspace{-3pt}

\item Used the \textbf{XGBClassifier (Scikit-learn)} to predict micropeptides to better identify, study, and alter protein structure.
\item Identified \textbf{20k} micro-peptides from \textbf{10TB+} of publicly available genome data.
\end{rSubsection} 

\vspace{-7pt}

\end{rSection}

%	EDUCATION

\begin{rSection}{EDUCATION}

{\textbf{Ph.D. in Crop Genetics}, University of Nottingham, UK} \hfill {Oct 2005 - Dec 2009}
\\
{\textbf{M.Sc. in Biotechnology}, G.B.P.U.A.T, India} \hfill {Jan 2001 - Mar 2003}
\\
{\textbf{B.Sc. in Agriculture}, A.N.G.R.A.U, India} \hfill {Aug 1996 - Jun 2000}

\end{rSection} 

%-------------------------------------------------- 

\end{document}
