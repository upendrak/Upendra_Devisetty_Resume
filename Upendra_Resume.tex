%%%%%%%%%%%%%%%%%%%%%%%%%%%%%%%%%%%%%%%%%
% Medium Length Professional CV
% LaTeX Template
% Version 2.0 (8/5/13)
%
% This template has been downloaded from:
% http://www.LaTeXTemplates.com
%
% Original author:
% Trey Hunner (http://www.treyhunner.com/)
%
% Important note:
% This template requires the resume.cls file to be in the same directory as the
% .tex file. The resume.cls file provides the resume style used for structuring the
% document.
%
%%%%%%%%%%%%%%%%%%%%%%%%%%%%%%%%%%%%%%%%%

%----------------------------------------------------------------------------------------
%	PACKAGES AND OTHER DOCUMENT CONFIGURATIONS 
%----------------------------------------------------------------------------------------
 
\documentclass{resume} % Use the custom resume.cls style 
\usepackage{hyperref}
\usepackage[left=0.4 in,top=0.15 in,right=0.4 in,bottom=0.3in]{geometry} % Document margins
\newcommand{\tab}[1]{\hspace{.2667\textwidth}\rlap{#1}}
\newcommand{\itab}[1]{\hspace{0em}\rlap{#1}}
\name{UPENDRA KUMAR DEVISETTY} % Your name 
\address{upendrakumar.devisetty@gmail.com \\ (520)-392-2482 \\ Seattle, WA} % Your address 
\address{\href{https://github.com/upendrak}{github.com/upendrak} \\ \href{http://linkedin.com/in/upendradevisetty}{linkedin.com/upendradevisetty} \\ \href{https://hub.docker.com/u/upendradevisetty}{docker.com/upendradevisetty}} % Your phone number and email
\begin{document}  

%----------------------------------------------------------------------------------------

\begin{rSection}{Skills} \itemsep -3pt  

{\textbf{Languages:} Python, R, SQL, Shell Scripting}  \\
{\textbf{Tools:} Keras, Scikit-learn, PySpark, Jupyter, Pandas, Matplotlib, Seaborn, Numpy, Git, Shiny, Flask, HTML, MySQL, AWS, Vagrant, Ansible, Openstack, Docker, Singularity and familiarity with Kubernetes} \\
{\textbf{Techniques:} Data cleaning and analysis, Regression, Classification, Clustering, Feature engineering, Modeling, Convolutional Neural Networks, Computer Vision, NLP, DevOps, HPC and GPU computing}
\end{rSection}  
 
%-----------------------------------------------------------------------------

\begin{rSection}{Experience } \itemsep -1pt        

\begin{rSubsection}{Data Science Fellow, Insight Data Science, Seattle, WA}{June 2019 - Present}{}    

\vspace{-3pt}

\item Built \href{https://disease-predictor.onrender.com}{Disease Predictor}, an image-based plant disease prediction app that help farmers accurately diagnose plant diseases.
\item Trained and validated a 24 layer CNN from 100K diseased and healthy plant leaf images using Keras.
\item Used GitHub and Docker to automatically manage building, testing and deploying the Flask web app on AWS.
\end{rSubsection} 

\vspace{-3pt}

\begin{rSubsection}{Data Science Instructor, Datacamp}
{Jan 2019 - Present}{}{}  %June 2016 - November 2016
\vspace{-3pt}
\item Designed and developed course content for \textbf{Big Data Fundamentals via PySpark} using Apache PySpark RDD, DataFrames, SparkSQL and SparkML. The course has been taken by over 6000 students to date.
\item Created \textbf{Top 10 programming languages for Data Scientists} project using PySpark to find out which programming languages do Data scientists use. The project uses Stack Overflow 2019 Developer Survey data which consists of ~100K responses from around the world.
\vspace{-3pt}
\end{rSubsection}  

\begin{rSubsection}{Science Informatician, CyVerse, AZ }{Jan 2016 - Present}{}    

\vspace{-3pt}

\item Optimized the process of bringing bioinformatics and data science tools and software into CyVerse using Docker.
\item Built custom bioinformatics computational workflows for processing 25 TB of Big and complex biological Data on the cloud.
\item Helped and trained researchers/students in design of experiments, analyze and plot data, run statistical software and interpret results.
\end{rSubsection} 

\vspace{-4pt}

\begin{rSubsection}{PostDoc researcher, OSU \& UCDavis }{Mar 2010 - Dec 2015}{}    

\vspace{-3pt}

\item Wrote custom scripts in Python, R and bash to pre-process the sequencing data and automated that using bash scripts.
\item Developed custom data analysis pipelines for analyzing 100 TB of genome data to understand how plants respond to environmental conditions.
\end{rSubsection} 
 
\end{rSection}

%	PROJECTS

\begin{rSection}{PROJECTS}

%------------------------------------------------
\begin{rSubsection}{Automatic Classification of Research Paper Abstracts}{Jan 2019 - Present}{} %Jun 2016 - Jul 2016   

\vspace{-3pt}

\item Used the NLTK toolkit to clean the data (stemming and lemmatization). 
\item Employed Scikit-learn's Logistic Regression and Keras's deep neural networks for classification of research paper abstracts. \end{rSubsection} 

\begin{rSubsection}{LNCPEP}{June 2018}{} %Jun 2016 - Jul 2016   

\vspace{-3pt}

\item Implemented \href{https://github.com/NCBI-Hackathons/LNCPEP}{LNCPEP}, a machine learning framework for identifying RNA data that encode micropeptides using Scikit-learn's XGBoost classification. 
\end{rSubsection} 

\end{rSection}


%	EDUCATION

\begin{rSection}{EDUCATION}

{\textbf{Ph.D. in Crop Genetics}, University of Nottingham, UK} \hfill {Oct 2005 - Dec 2009}
\\
{\textbf{M.Sc. in Biotechnology}, G.B.P.U.A.T, India} \hfill {Jan 2001 - Mar 2003}
\\
{\textbf{B.Sc. in Agriculture}, A.N.G.R.A.U, India} \hfill {Aug 1996 - Jun 2000}

\end{rSection} 

%-------------------------------------------------- 

\end{document}
